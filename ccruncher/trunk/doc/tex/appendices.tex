
%***************************************************************************
%
% CreditCruncher - A portfolio credit risk valorator
% Copyright (C) 2004 Gerard Torrent
%
% This program is free software; you can redistribute it and/or
% modify it under the terms of the GNU General Public License
% as published by the Free Software Foundation; either version 2
% of the License.
%
% This program is distributed in the hope that it will be useful,
% but WITHOUT ANY WARRANTY; without even the implied warranty of
% MERCHANTABILITY or FITNESS FOR A PARTICULAR PURPOSE.  See the
% GNU General Public License for more details.
%
% You should have received a copy of the GNU General Public License
% along with this program; if not, write to the Free Software
% Foundation, Inc., 59 Temple Place - Suite 330, Boston, MA 02111-1307, USA.
%
%
% appendices.tex - TeX documentation file
% --------------------------------------------------------------------------
%
% 2005/01/22 - Gerard Torrent [gerard@fobos.generacio.com]
%   . initial release
%
%***************************************************************************

\chapter{Ap\'endices}
\label{sec:apendixes}

%---------------------------------------------------------------------------
\section{Conceptos b\'asicos de estad\'istica}
\label{apendix:stats}

\paragraph{Definici\'on.} Llamamos \emph{funci\'on de distribuci\'on}
\index{Funci\'on de distribuci\'on} o cdf de la variable aleatoria $X$ a la
funci\'on $F$ que cumple:
\begin{displaymath}
F(x) = P(X \leq x)
\end{displaymath}

\paragraph{Definici\'on.} Llamamos \emph{funci\'on de probabilidad}
\index{Funci\'on de probabilidad} o \emph{densidad}\index{Densidad} o pdf
 de la variable aleatoria $X$ a la funci\'on $f$ que cumple:
\begin{displaymath}
F(x) = \int_{-\infty}^{x} f(t) dt
\end{displaymath}

\paragraph{Proposici\'on.} Sea $X$ una variable aleatoria continua con
funci\'on de densidad $f_X(x)$. La funci\'on de densidad de $Y=H(X)$ siendo 
$H(.)$ mon\'otona (estrictamente creciente o estrictamente decreciente) es:
\begin{displaymath}
f_Y(y) = f_X(H^{-1}(y))\cdot \left| \frac{d(H^{-1}(y))}{dy} \right|
\end{displaymath}

\paragraph{Esperanza.} Definimos la \emph{esperanza}\index{Esperanza} de una
variable aleatoria discreta de la forma siguiente:
\begin{displaymath}
E(X) = \sum_{i} i \cdot P(X=i)
\end{displaymath}
En el caso de una variable aleatoria continua con funci\'on de distribuci\'on 
$f(x)$ la esperanza se expresa como:
\begin{displaymath}
E(X) = \int_{-\infty}^{\infty} x \cdot f(x) dx
\end{displaymath}

\paragraph{Varianza.} Definimos la \emph{varianza}\index{Varianza} de una variable
aleatoria discreta de la forma siguiente:
\begin{displaymath}
Var(X) = \sum_{i} (i-E(X))^2 \cdot P(X=i)
\end{displaymath}
En el caso de una variable aleatoria continua con funci\'on de distribuci\'on 
$f(x)$ la varianza se expresa como:
\begin{displaymath}
Var(X) = \int_{-\infty}^{\infty} (x-E(X))^2 \cdot f(x) dx
\end{displaymath}

\paragraph{Covarianza.} Definimos la \emph{covarianza}\index{Covarianza} entre dos
variables aleatorias $X$ e $Y$ de la forma siguiente:
\begin{displaymath}
Cov(X,Y) = E(X \cdot Y) - E(X) \cdot E(Y)
\end{displaymath}

\paragraph{Correlaci\'on.} Definimos la \emph{correlaci\'on}\index{Correlaci\'on},
$\rho$, entre dos variables aleatorias $X$ e $Y$ de la forma siguiente:
\begin{displaymath}
\rho_{X,Y} = \frac{Cov(X,Y)}{\sqrt{Var(x) \cdot Var(Y)}}
\end{displaymath}

\paragraph{Proposici\'on.} Sea $X$ una variable aleatoria continua con
funci\'on de densidad $f(x)$ y $H(x)$ una funci\'on diferenciable. Entonces:
\begin{displaymath}
E(H(X)) = \int_{-\infty}^{\infty} H(x) \cdot f(x) dx
\end{displaymath}


%---------------------------------------------------------------------------

\section{La variable aleatoria de Bernoulli}

\paragraph{Definici\'on.} La variable aleatoria discreta \emph{Bernouilli}
\index{Bernouilli}, $X$, se utiliza para modelar fen\'omenos que solamente pueden
tomar dos estados, $0$ y $1$, con probabilidades $p$ y $(1-p)$ respectivamente.
La notaremos como $X \sim Ber(p)$:
\begin{displaymath}
P(X=0) = (1 - p) \qquad   P(X=1) = p \qquad p \in [0,1]
\end{displaymath}
 
\paragraph{Esperanza.} La esperanza de una variable aleatoria Bernouilli $X \sim Ber(p)$ 
es $p$. Este resultado es la aplicaci\'on directa de la definici\'on de esperanza 
para una variable aleatoria discreta:
\begin{displaymath}
E(X) = \sum_{i} i \cdot P(X=i) = 1 \cdot p + 0 \cdot (1-p) = p
\end{displaymath}

\paragraph{Varianza.} La varianza de una variable aleatoria Bernouilli $X \sim Ber(p)$ 
es $p \cdot (1-p)$. Este resultado es la aplicaci\'on directa de la definici\'on 
de varianza para una variable aleatoria discreta:
\begin{displaymath}
Var(X)= \sum_{i} (i-E(X))^2 \cdot P(X=i) = (1-p)^2 \cdot p + (-p)^2 \cdot (1-p) = p \cdot (1-p)
\end{displaymath}
 
\paragraph{Simulaci\'on.} La simulaci\'on de una variable Bernouilli 
$X \sim Ber(p)$ la realizamos de la siguiente forma:
\begin{displaymath}
x= \left\{
\begin{array}{cc}
0 & u \in [0,1-p) \cr
1 & u \in [1-p,1]
\end{array}
\right.
\qquad u \sim U[0,1]
\end{displaymath}

%---------------------------------------------------------------------------

\section{La variable aleatoria Binomial}

\paragraph{Definici\'on.} La suma de $n$ variables aleatorias Bernoulli, $Ber(p)$,  
independientes e id\'enticamente distribuidas es una variable aleatoria discreta, 
$X$ que llamamos \emph{Binomial}\index{Binomial}, $X \sim B(n,p)$.
\begin{displaymath}
P(X=k) = {n \choose k} \cdot p^k \cdot (1-p)^{n-k} \qquad {n \choose k} = \frac{n!}{k! \cdot (n-k)!}
\qquad k \in \{0, \cdots, n\}
\end{displaymath}

\paragraph{Esperanza.} La esperanza de una variable aleatoria Binomial 
$X \sim B(n,p)$ es:
\begin{displaymath}
E(X) = n \cdot p
\end{displaymath}

\paragraph{Varianza.} La varianza de una variable aleatoria Binomial 
$X \sim B(n,p)$ es:
\begin{displaymath}
Var(X)= n \cdot p \cdot (1-p)
\end{displaymath}

\paragraph{Proposici\'on.} El Teorema Central del L\'imite nos permite, en el
caso de $n$ grandes, aproximar la distribuci\'on discreta Binomial por una 
distribuci\'on continua Normal:
\begin{displaymath}
B(n,p) \approx N\left(n p, n p (1-p)\right)
\end{displaymath}


%---------------------------------------------------------------------------

\section{La variable aleatoria Normal}

\paragraph{Definici\'on.} Decimos que una variable aleatoria continua $Z$ es 
una \emph{Normal}\index{Normal} con media $\mu$ y desviaci\'on est\'andar $\sigma$, 
$Z \sim N(\mu, \sigma^2)$ si tiene la siguiente funci\'on de densidad:
\begin{displaymath}
f(x) = \frac{1}{\sigma \sqrt{2 \pi}} e^{-\frac{(x-\mu)^2}{2 \sigma^2}}
\end{displaymath}

\paragraph{Esperanza.} La esperanza de una variable aleatoria Normal 
$X \sim N(\mu,\sigma^2)$ es:
\begin{displaymath}
E(X) = \mu
\end{displaymath}

\paragraph{Varianza.} La varianza de una variable aleatoria Normal 
$X \sim N(\mu,\sigma^2)$ es:
\begin{displaymath}
Var(X)= \sigma^2
\end{displaymath}

\paragraph{Simulaci\'on.} Para la generaci\'on de una realizaci\'on, $z$, de 
una variable aleatoria normal $Z \sim N(\mu, \sigma^2)$ utilizamos el siguiente 
algoritmo:
\begin{displaymath}
z = \mu + \sigma\cdot \sqrt{-2 ln(u_1)} \cdot cos(2 \pi \cdot u_2)
\qquad u_1, u_2 \sim U[0,1]
\end{displaymath}

\paragraph{Definici\'on.} Decimos que una variable aleatoria continua 
$n$-dimensional, $Z$, es una \emph{Normal}\index{Normal} con media $\vec{\mu}$
y matriz de covarianza $\Sigma$, $Z \sim N(\vec{\mu}, \Sigma)$, si tiene la
siguiente funci\'on de densidad:
\begin{displaymath}
f(\vec{x}) = \frac{1}{\sqrt{(2 \pi)^n \left| \Sigma \right|}} 
exp\left\{-\frac{1}{2}\left(\vec{x}-\vec{\mu})^{\top} \Sigma^{-1} (\vec{x}-\vec{\mu}) \right)\right\}
\end{displaymath}
donde $\left|\Sigma\right|$ es el determinante de la matriz de covarianzas 
$\Sigma$, y $\Sigma^{-1}$ es la inversa de la matriz $\Sigma$.

\paragraph{Simulaci\'on.} Para la generaci\'on de una realizaci\'on, $\vec{z}$, 
de una variable aleatoria normal $Z \sim N(\vec{\mu}, \Sigma)$ utilizamos el 
siguiente algoritmo:
\begin{displaymath}
\vec{z} = \vec{\mu} + \Sigma^{1/2} \vec{x}
\qquad x_i \sim N[0,1]
\end{displaymath}
La matriz $\Sigma^{1/2}$ la calculamos usando el algoritmo de Cholesky. Sabemos 
que existe soluci\'on debido a que $\Sigma$ es definida positiva por tratarse
de una matriz de covarianzas.

%---------------------------------------------------------------------------

\section{C\'alculo de la raiz de una matriz}
\label{apendix:sqrtmat}

\paragraph{Definici\'on.}
Diremos que 2 matrices $A$ y $B$ de orden $n$ son \emph{semejantes}
\index{Matrices semejantes} si existe una matriz, $P$, de orden $n$ con
$det(P) \neq 0$ tal que $B = P^{-1} \cdot A \cdot P$.


\paragraph{Proposici\'on.} Si dos matrices $A$ y $B$ son semejantes 
($B = P^{-1} \cdot A \cdot P$) entonces:
\begin{displaymath}
det(A) = det(B)
\end{displaymath}
\begin{displaymath}
B^n = P^{-1} \cdot A^{n} \cdot P
\end{displaymath}

\paragraph{Definici\'on.} 
Diremos que una matriz $A$ de orden $n$ es \emph{diagonalizable}
\index{Matriz diagonalizable} si es semejante a una matriz diagonal $D$, o sea,
$A = P^{-1} \cdot D \cdot P$ siendo $det(D) \neq 0$.

\paragraph{Proposici\'on.} 
Para que una matriz $A$ sea diagonalizable es necesario y suficiente que:
\begin{itemize}
\item Los valores propios de $A$ sean todos reales
\item Los $n$ vectores propios de $A$ sean independientes
\end{itemize}

\paragraph{Proposici\'on.}
Si una matriz $A$ es diagonalizable ($A = P^{-1} \cdot D \cdot P$) entonces: 
\begin{itemize}
\item $D$ es una matriz diagonal compuesta por los valores propios de la matriz $A$
\item $P$ es la matriz formada por los vectores propios de la matriz $A$
\end{itemize}

\paragraph{Resultado.}
Sea $A$ la ra\'iz $n$-esima de una matriz diagonalizable $B$. Entonces:
\begin{displaymath}
A^n = B = P^{-1} \cdot D \cdot P 
\Longrightarrow  
A = \sqrt[n]{B} = P^{-1} \cdot \sqrt[n]{D} \cdot P
\end{displaymath} 

%---------------------------------------------------------------------------

\section{Algoritmo de la c\'opula normal}

\index{C\'opula normal} Sea $\Sigma_1$ la matriz de correlaci\'on a cumplir
por la c\'opula. Observemos que se trata tambi\'en de la matriz de covarianzas
al valer $1$ los elementos de la diagonal.
\begin{displaymath}
\Sigma_1 = \left( 
\begin{array}{cccc}
1          & \rho_{12} & \ldots & \rho_{1n} \cr
\rho_{21} & 1          & \ldots & \rho_{2n} \cr
\vdots    & \vdots    & \ddots & \vdots   \cr
\rho_{n1} & \rho_{n2} & \ldots & 1
\end{array}
\right)
\end{displaymath}

\paragraph{Paso 1.} Creamos la matrix de covarianzas $\Sigma_2$ transformando 
la matriz $\Sigma_1$ componente a componente:
\begin{displaymath}
\Sigma_2 = \left( 
\begin{array}{cccc}
2 sin(\frac{\pi}{6})           & 2 sin(\rho_{12} \frac{\pi}{6}) & \ldots & 2 sin(\rho_{1n} \frac{\pi}{6})\cr
2 sin(\rho_{21} \frac{\pi}{6}) & 2 sin(\frac{\pi}{6})           & \ldots & 2 sin(\rho_{2n} \frac{\pi}{6})\cr
\vdots                          & \vdots                          & \ddots  & \vdots   \cr
2 sin(\rho_{n1} \frac{\pi}{6}) & 2 sin(\rho_{n2} \frac{\pi}{6}) & \ldots & 2 sin(\frac{\pi}{6})
\end{array}
\right)
\end{displaymath}
Observamos que la matriz $\Sigma_2$ vuelve a tener los elementos de la
diagonal iguales a $1$ debido a que $2 sin(\frac{\pi}{6}) = 1$.

\paragraph{Paso 2.} A la matriz $\Sigma_2$ le aplicamos el algoritmo de 
Cholesky para obtener la matrix triangular inferior $B$ cumpliendo 
$\Sigma_2 = B \cdot B^{\top}$:
\begin{displaymath}
B = 
\left(
\begin{array}{cccc}
b_{11}   & 0        & \ldots & 0       \cr
b_{21}   & b_{22}   & \ldots & 0       \cr
\vdots  & \vdots  & \ddots & \vdots \cr
b_{n1}   & b_{n2}   & \ldots & b_{nn}
\end{array}
\right)
\end{displaymath}

\paragraph{Paso 3.} Simulamos $n$ variables aleatorias $N(0,1)$ independientes:
\begin{displaymath}
\vec{Y}^{\top}=(Y_1, \cdots, Y_n)^{\top} \qquad Y_k \sim N(0,1) \textrm{ independientes}
\end{displaymath}

\paragraph{Paso 4.} Simulamos una variable n-dimensional $Z \sim N(\vec{0}, \Sigma_2)$
haciendo:
\begin{displaymath}
\vec{Z}^{\top} = 
\left(
\begin{array}{c}
Z_1 \cr
\vdots \cr
Z_n
\end{array}
\right) 
=
\left(
\begin{array}{cccc}
b_{11}   & 0        & \ldots & 0       \cr
b_{21}   & b_{22}   & \ldots & 0       \cr
\vdots  & \vdots  & \ddots & \vdots \cr
b_{n1}   & b_{n2}   & \ldots & b_{nn}
\end{array}
\right)
\left(
\begin{array}{c}
Y_1 \cr
\vdots \cr
Y_n
\end{array}
\right) 
 = B \cdot \vec{Y}^{\top}
\end{displaymath}

\paragraph{Paso 5.} Finalmente obtenemos la simulaci\'on de la c\'opula, $\vec{X}$.
\begin{displaymath}
\vec{X}^{\top} = (X_1, \cdots, X_n)^{\top} = (\Phi(Z_1), \cdots, \Phi(Z_n))^{\top}
\end{displaymath}
donde $\Phi(x)$ es la funci\'on de distribuci\'on de la Normal est\'andar
\begin{displaymath}
\Phi(x) = \int_{-\infty}^{x} \frac{1}{\sqrt{2 \pi}} e^{-\frac{t^2}{2}} dt
\end{displaymath}

%---------------------------------------------------------------------------

\section{Descomposici\'on de Cholesky de una matriz en bloques}

Dada una matriz cuadrada, sim\'etrica y definida positiva, $A$, el algoritmo
de Cholesky\index{Algoritmo de Cholesky} realiza la siguiente descomposici\'on:
\begin{displaymath}
U^{\top} \cdot U = A
\end{displaymath}
donde $U$ es una matriz triangular superior (por tanto, $U^{\top}$ es una matriz
triangular inferior). Una descripci\'on e implementaci\'on del algoritmo puede
encontrarse en \emph{Numerical Recipes in C}\footnote{http://www.nr.com}.
\newline
\newline
Si intentamos realizar la descomposici\'on de Cholesky de una matriz de correlaci\'on
entre clientes de una cartera de $50.000$ clientes nos encontraremos con problemas
de tama\~no (la matriz ocupar\'a 19 Gb. de memoria) y n\'umero de operaciones al
multiplicar la matriz por un vector ($2.500.000.000$ multiplicaciones).
\newline
\newline
Modificaremos el algoritmo de Cholesky para aprovechar el hecho que la matriz
de correlaci\'on entre clientes es una matriz en bloques con unos en la diagonal.
Veamos un ejemplo de una cartera de $7$ clientes con dos sectores:
\begin{displaymath}
A = \left(
\begin{array}{cccc|ccc}
1   & 0.5 & 0.5 & 0.5 & 0.1 & 0.1 & 0.1 \cr
0.5 & 1   & 0.5 & 0.5 & 0.1 & 0.1 & 0.1 \cr
0.5 & 0.5 & 1   & 0.5 & 0.1 & 0.1 & 0.1 \cr
0.5 & 0.5 & 0.5 & 1   & 0.1 & 0.1 & 0.1 \cr
\hline
0.1 & 0.1 & 0.1 & 0.1 & 1   & 0.3 & 0.3 \cr
0.1 & 0.1 & 0.1 & 0.1 & 0.3 & 1   & 0.3 \cr
0.1 & 0.1 & 0.1 & 0.1 & 0.3 & 0.3 & 1
\end{array}
\right)
\end{displaymath}
Realizamos la descomposici\'on de Cholesky:
\begin{displaymath}
U = \left(
\begin{array}{cccc|ccc}
 1.00000 & 0.50000 & 0.50000 & 0.50000 & 0.10000 & 0.10000 & 0.10000 \cr
 0       & 0.86603 & 0.28868 & 0.28868 & 0.05774 & 0.05774 & 0.05774 \cr
 0       & 0       & 0.81650 & 0.20412 & 0.04082 & 0.04082 & 0.04082 \cr
 0       & 0       & 0       & 0.79057 & 0.03162 & 0.03162 & 0.03162 \cr
\hline
 0       & 0       & 0       & 0       & 0.99197 & 0.28630 & 0.28630 \cr
 0       & 0       & 0       & 0       & 0       & 0.94975 & 0.21272 \cr
 0       & 0       & 0       & 0       & 0       & 0       & 0.92563
\end{array}
\right)
\end{displaymath}
Observamos que $U$ contiene elementos repetidos. Guardaremos la matriz $U$
en memoria de la forma siguiente:
\begin{displaymath}
U = \left|
\begin{array}{c|cc}
 1.00000 & 0.50000 & 0.10000 \cr
 0.86603 & 0.28868 & 0.05774 \cr
 0.81650 & 0.20412 & 0.04082 \cr
 0.79057 & 0       & 0.03162 \cr
 0.99197 & 0       & 0.28630 \cr
 0.94975 & 0       & 0.21272 \cr
 0.92563 & 0       & 0
\end{array}
\right|
\end{displaymath}
o sea, para cada fila guardamos el valor de la diagonal y el valor de
cada sector. El tama\~no en memoria ahora pasa a ser $N \times (M+1)$
donde $N$ es el n\'umero de clientes y $M$ el n\'umero de sectores.
\newline
\newline
Para reducir el n\'umero de operaciones realizadas al multiplicar la
matriz por un vector aprovechamos que la matriz $U$ tiene elementos
repetidos. Veamos un ejemplo:
\begin{displaymath}
\begin{array}{l}
(U \cdot x)_2 =  0.86603 \cdot x_2 + 0.28868 \cdot x_3 + 0.28868 \cdot x_4 + \cr
                 0.05774 \cdot x_5 + 0.05774 \cdot x_6 + 0.05774 \cdot x_7 \cr
              = 0.86603 \cdot x_2 + 2\cdot 0.28868 \cdot (x_3 + x_4) + 3 \cdot 0.05774 \cdot (x_5 + x_6 + x_7)
\end{array}
\end{displaymath}

Con estas dos consideraciones se obtiene una algoritmo de Cholesky para matrices
en bloques con uso de memoria y n\'umero de operaciones (al multiplicar por un
vector) del orden $N \times (M+1)$ en vez de orden $N^2$.
