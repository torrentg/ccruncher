
%***************************************************************************
%
% CreditCruncher - A portfolio credit risk valorator
% Copyright (C) 2004 Gerard Torrent
%
% This program is free software; you can redistribute it and/or
% modify it under the terms of the GNU General Public License
% as published by the Free Software Foundation; either version 2
% of the License.
%
% This program is distributed in the hope that it will be useful,
% but WITHOUT ANY WARRANTY; without even the implied warranty of
% MERCHANTABILITY or FITNESS FOR A PARTICULAR PURPOSE.  See the
% GNU General Public License for more details.
%
% You should have received a copy of the GNU General Public License
% along with this program; if not, write to the Free Software
% Foundation, Inc., 59 Temple Place - Suite 330, Boston, MA 02111-1307, USA.
%
%
% appendices.tex - TeX documentation file
% --------------------------------------------------------------------------
%
% 2005/01/22 - Gerard Torrent [gerard@fobos.generacio.com]
%   . initial release
%
%***************************************************************************

\chapter{Ap\'endices}
\label{sec:apendixes}

%---------------------------------------------------------------------------

\section{Conceptos b\'asicos de estad\'istica}


%---------------------------------------------------------------------------

\section{C\'alculo de la raiz de una matriz}

\paragraph{Definici\'on.}
Diremos que 2 matrices $A$ y $B$ de orden $n$ son semejantes si existe una 
matriz, $P$, de orden $n$ con $det(P) \neq 0$ tal que 
$B = P^{-1} \cdot A \cdot P$.


\paragraph{Proposici\'on.} Si dos matrices $A$ y $B$ son semejantes 
($B = P^{-1} \cdot A \cdot P$) entonces:
\begin{displaymath}
det(A) = det(B)
\end{displaymath}
\begin{displaymath}
B^n = P^{-1} \cdot A^{n} \cdot P
\end{displaymath}

\paragraph{Definici\'on.} 
Diremos que Una matriz $A$ de orden $n$ es diagonalizable si es semejante a una 
matriz diagonal $D$, o sea, $A = P^{-1} \cdot D \cdot P$ siendo $det(D) \neq 0$.

\paragraph{Proposici\'on.} 
Para que una matriz $A$ sea diagonalizable es necesario y suficiente que:
\begin{itemize}
\item Los valores propios de $A$ sean todos reales
\item Los $n$ vectores propios de $A$ sean independientes
\end{itemize}

\paragraph{Proposici\'on.}
Si una matriz $A$ es diagonalizable ($A = P^{-1} \cdot D \cdot P$) entonces: 
\begin{itemize}
\item $D$ es una matriz diagonal compuesta por los valores propios de la matriz $A$
\item $P$ es la matriz formada por los vectores propios de la matriz $A$
\end{itemize}

\paragraph{Resultado.}
Sea $A$ la ra\'iz $n$-esima de una matriz diagonalizable $B$. Entonces:
\begin{displaymath}
A^n = B = P^{-1} \cdot D \cdot P 
\Longrightarrow  
A = \sqrt[n]{B} = P^{-1} \cdot \sqrt[n]{D} \cdot P
\end{displaymath} 

%---------------------------------------------------------------------------

\section{La variable aleatoria de Bernoulli}

\subsection{Definici\'on y propiedades}

\subsection{Simulaci\'on}

TODO: ampliarla al caso x1, ..., xn. simulacion usando uniforme [0,1] + etc.

%---------------------------------------------------------------------------

\section{La variable aleatoria normal}

\subsection{Definici\'on y propiedades}

\begin{displaymath}
P(X \leq x) = \Phi(x) = \int_{-\infty}^{x} \frac{e^{-t^2}}{\sqrt{2 \pi}} dt
\end{displaymath}

\subsection{Simulaci\'on}

Para la generaci\'on de una realizaci\'on, $z$, de una variable aleatoria normal  
$Z \sim N(\mu, \sigma)$ utilizamos el siguiente algoritmo:

\begin{displaymath}
z = \mu + \sigma\cdot \sqrt{-2 \cdot ln(u[0,1])} \cdot cos(2 \cdot \pi \cdot u[0,1])
\end{displaymath}

\noindent donde $u[0,1]$ son realizaciones de una variable aleatoria uniforme 
en el intervalo $[0,1]$.
