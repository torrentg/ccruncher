
%***************************************************************************
%
% CreditCruncher - A portfolio credit risk valorator
% Copyright (C) 2004 Gerard Torrent
%
% This program is free software; you can redistribute it and/or
% modify it under the terms of the GNU General Public License
% as published by the Free Software Foundation; either version 2
% of the License.
%
% This program is distributed in the hope that it will be useful,
% but WITHOUT ANY WARRANTY; without even the implied warranty of
% MERCHANTABILITY or FITNESS FOR A PARTICULAR PURPOSE.  See the
% GNU General Public License for more details.
%
% You should have received a copy of the GNU General Public License
% along with this program; if not, write to the Free Software
% Foundation, Inc., 59 Temple Place - Suite 330, Boston, MA 02111-1307, USA.
%
%
% introduction.tex - TeX documentation file
% --------------------------------------------------------------------------
%
% 2004/12/04 - Gerard Torrent [gerard@fobos.generacio.com]
%   . initial release
%
%***************************************************************************

\chapter{Introducci\'on}
\label{sec:introduction}

Este documento contiene la descripci\'on del m\'etodo de valoraci\'on del riesgo
de cr\'edito implementado por el programa CreditCruncher. Para su lectura no se 
presuponen conocimientos avanzados de matem\'aticas o finanzas. En caso de 
encontrar un error, sugerir mejoras o no entender alg\'un punto,
no dude en ponerse en contacto con el equipo de desarrollo de 
CreditCruncher\footnote{http://www.generacio.com/ccruncher/} que tendr\'a en 
cuenta sus aportaciones para futuras versiones de este documento.


\section{Acerca de CreditCruncher}

La valoraci\'on del riesgo de cr\'edito no es un tema cerrado, muestra de ello 
es la multitud de m\'etodos que existen para su valoraci\'on. Se recomienda la 
lectura del excelente art\'iculo \emph{Different strokes} \cite{Risk:Dif_Str} 
donde se exponen los principales modelos de valoraci\'on del riesgo de cr\'edito 
y sus caracter\'isticas. 
\newline
\newline
CreditCruncher valora el riesgo de impago de una cartera de cr\'editos usando la 
t\'ecnica de simulaci\'on Monte Carlo. Pretende ofrecer un m\'etodo de valoraci\'on 
del riesgo de cr\'edito totalmente documentado y soportado por una implementaci\'on 
libre y gratuita. Pertenece a la fam\'ilia de m\'etodos tipo 
CreditMetrics\footnote{http://www.riskmetrics.com/}.
\newline
\newline
La mayor\'ia de conceptos y explicaciones que pueden 
encontrarse en este documento han sido extraidas o inspiradas en el excelente 
documento \emph{CreditMetrics - Technical Document} \cite{CreditMetrics:Tech_Doc}.
Puede usarse el art\'iculo \emph{Probability models of credit risk} \cite{cbs:glasser} 
como una introducci\'on corta y clara.


\section{Organizaci\'on del contenido}

Se ha organizado el contenido en cuatro secciones principales y un conjunto de 
anexos.

\paragraph{Formulaci\'on del problema.} Contiene la descripci\'on del problema
que se pretende resolver y se introducen los elementos y propiedades considerados 
claves para la posterior resoluci\'on. La lectura de este apartado es necesaria para
entender los elementos del fichero de entrada de datos del programa.

\paragraph{Resoluci\'on del problema.} Se exponen los elementos usados para 
resolver el problema y se detalla la estructura del m\'etodo de resoluci\'on.
La lectura de este apartado es necesaria para la interpretaci\'on de los 
resultados proporcionados por el programa.

\paragraph{Implementaci\'on de la soluci\'on.} Se explican los detalles de
la implementaci\'on. La lectura de este apartado es necesaria para entender 
alguno de los apartados del fichero de entrada de datos del programa as\'i como 
para la interpretaci\'on de los resultados proporcionados por este.

\paragraph{Ejemplos.} Conjunto de ejemplos representativos resueltos con
CreditCruncher. Los ficheros de entrada de los ejemplos se incluien en la
la aplicaci\'on. La lectura de este apartado, junto con los ficheros de ejemplo 
pueden ayudarle en la creaci\'on de sus primeros ficheros de entrada.

\paragraph{Anexos.} Contienen elementos necesarios para la comprensi\'on del 
contenido de las secciones principales, pero que su inclusi\'on en estas 
oscurecer\'ia la explicaci\'on. \\

No se demuestran los enunciados que puedan ser encontrados
en los libros de matem\'aticas de grado medio o superior. Las referencias 
bibliogr\'aficas que se incluien no son para hacer bonito, pueden ayudarle en
la comprensi\'on de lo expuesto en este documento.


