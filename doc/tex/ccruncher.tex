%%%%%%%%%%%%%%%%%%%%%%%%%%%%%%%%%%%%%%%%%
% The Legrand Orange Book
% LaTeX Template
% Version 1.1 (11/4/13)
%
% This template has been downloaded from:
% http://www.LaTeXTemplates.com
%
% Original author:
% Mathias Legrand (legrand.mathias@gmail.com)
%
% License:
% CC BY-NC-SA 3.0 (http://creativecommons.org/licenses/by-nc-sa/3.0/)
%
% Compiling this template:
% This template uses biber for its bibliography and makeindex for its index.
% This means that to update the bibliography and index in this template you
% will need to run the following sequence of commands in the template
% directory:
%
% 1) pdflatex main
% 2) makeindex main.idx -s StyleInd.ist
% 3) biber main
% 4) pdflatex main
%
% This template also uses a number of packages which may need to be
% updated to the newest versions for the template to compile. It is strongly
% recommended you update your LaTeX distribution if you have any
% compilation errors.
%
% Important note:
% Chapter heading images should have a 2:1 width:height ratio,
% e.g. 920px width and 460px height.
%
%%%%%%%%%%%%%%%%%%%%%%%%%%%%%%%%%%%%%%%%%

%----------------------------------------------------------------------------------------
%	PACKAGES AND OTHER DOCUMENT CONFIGURATIONS
%----------------------------------------------------------------------------------------

\documentclass[11pt,fleqn]{book} % Default font size and left-justified equations

\usepackage[top=3cm,bottom=3cm,left=3.2cm,right=3.2cm,headsep=10pt,a4paper]{geometry} % Page margins

\usepackage{xcolor} % Required for specifying colors by name
\definecolor{ocre}{RGB}{243,102,25} % Define the orange color used for highlighting throughout the book

% Font Settings
\usepackage{avant} % Use the Avantgarde font for headings
%\usepackage{times} % Use the Times font for headings
\usepackage{mathptmx} % Use the Adobe Times Roman as the default text font together with math symbols from the Sym­bol, Chancery and Com­puter Modern fonts
\usepackage{microtype} % Slightly tweak font spacing for aesthetics
\usepackage[utf8]{inputenc} % Required for including letters with accents
\usepackage[T1]{fontenc} % Use 8-bit encoding that has 256 glyphs

% Bibliography
\usepackage[style=alphabetic,sorting=nyt,sortcites=true,autopunct=true,babel=hyphen,hyperref=true,abbreviate=false,backref=true,backend=biber]{biblatex}
\addbibresource{bibliography.bib} % BibTeX bibliography file
\defbibheading{bibempty}{}

% Index
\usepackage{calc} % For simpler calculation - used for spacing the index letter headings correctly
\usepackage{makeidx} % Required to make an index
\makeindex % Tells LaTeX to create the files required for indexing

% added by GTG
\usepackage{verbatim}
\usepackage{parskip}
%\setlength{\parindent}{0ex}
%\setlength{\parskip}{\baselineskip}

%----------------------------------------------------------------------------------------

\input{structure} % Insert the commands.tex file which contains the majority of the structure behind the template

\begin{document}

%----------------------------------------------------------------------------------------
%	TITLE PAGE
%----------------------------------------------------------------------------------------

\begingroup
\thispagestyle{empty}
\AddToShipoutPicture*{\put(6,5){\includegraphics[scale=1]{background}}} % Image background
\centering
\vspace*{8cm}
\par\normalfont\fontsize{35}{35}\sffamily\selectfont
Credit Risk Modeling \par % Book title
\par\normalfont\fontsize{25}{25}\sffamily\selectfont
Using CCruncher\par
\vspace*{2cm}
{\Huge Gerard Torrent}\par % Author name
\endgroup

%----------------------------------------------------------------------------------------
%	COPYRIGHT PAGE
%----------------------------------------------------------------------------------------

\newpage
~\vfill
\thispagestyle{empty}

\noindent Copyright \copyright\ 2013 Gerard Torrent\\ % Copyright notice

\noindent \textsc{Version 2.3 [R1023]}\\ 

\noindent \textsc{www.ccruncher.net}\\ % URL

\noindent 
This work is licensed to the public under Creative Commons Attribution-ShareAlike 3.0 Unported License. 
To view a copy of this license visit \url{http://creativecommons.org/licenses/by-sa/3.0/}.

\begin{comment}
Licensed under the Creative Commons Attribution-ShareAlike 3.0 Unported License 
(the ``License''). You may not use this file except in compliance with the License. 
You may obtain a copy of the License at \url{http://creativecommons.org/licenses/by-sa/3.0/}. 
Unless required by applicable law or agreed to in writing, software distributed under 
the License is distributed on an \textsc{``AS IS'' BASIS, WITHOUT WARRANTIES OR CONDITIONS 
OF ANY KIND}, either express or implied. See the License for the specific language governing 
permissions and limitations under the License.\\ % License information
\end{comment}

%\noindent \textit{First printing, March 2013} % Printing/edition date

%----------------------------------------------------------------------------------------
%	TABLE OF CONTENTS
%----------------------------------------------------------------------------------------

\chapterimage{chapter_head_1.pdf} % Table of contents heading image

\pagestyle{empty} % No headers

\tableofcontents % Print the table of contents itself

\cleardoublepage % Forces the first chapter to start on an odd page so it's on the right

\pagestyle{fancy} % Print headers again

%----------------------------------------------------------------------------------------
%	CHAPTER 1
%----------------------------------------------------------------------------------------

\chapterimage{chapter_head_2.pdf} % Chapter heading image

\chapter{Text Chapter}

\section{Paragraphs of Text}\index{Paragraphs of Text}

\lipsum[1-7] % Dummy text

%------------------------------------------------

\section{Citation}\index{Citation}

This statement requires citation \cite{book_key}; this one is more specific \cite[122]{article_key}.

%------------------------------------------------

\section{Lists}\index{Lists}

Lists are useful to present information in a concise and/or ordered way\footnote{Footnote example...}.

\subsection{Numbered List}\index{Lists!Numbered List}

\begin{enumerate}
\item The first item
\item The second item
\item The third item
\end{enumerate}

\subsection{Bullet Points}\index{Lists!Bullet Points}

\begin{itemize}
\item The first item
\item The second item
\item The third item
\end{itemize}

\subsection{Descriptions and Definitions}\index{Lists!Descriptions and Definitions}

\begin{description}
\item[Name] Description
\item[Word] Definition
\item[Comment] Elaboration
\end{description}

%----------------------------------------------------------------------------------------
%	CHAPTER 2
%----------------------------------------------------------------------------------------

\chapter{In-text Elements}

\section{Theorems}\index{Theorems}

This is an example of theorems.

\subsection{Several equations}\index{Theorems!Several Equations}

\begin{theorem}
In $E=\mathbb{R}^n$ all norms are equivalent. It has the properties:
\begin{align}
& \big| ||\mathbf{x}|| - ||\mathbf{y}|| \big|\leq || \mathbf{x}- \mathbf{y}||\\
&  ||\sum_{i=1}^n\mathbf{x}_i||\leq \sum_{i=1}^n||\mathbf{x}_i||\quad\text{where $n$ is a finite integer}
\end{align}
\end{theorem}

\subsection{Single Line}\index{Theorems!Single Line}

\begin{theorem}
A set $\mathcal{D}(G)$ in dense in $L^2(G)$, $|\cdot|_0$. 
\end{theorem}

%------------------------------------------------

\section{Definitions}\index{Definitions}

This is an example of a definition. A definition could be mathematical or it could define a concept.

\begin{definition}[Definition name]
Given a vector space $E$, a norm on $E$ is an application, denoted $||\cdot||$, $E$ in $\mathbb{R}^+=[0,+\infty[$ such that:
\begin{align}
& ||\mathbf{x}||=0\ \Rightarrow\ \mathbf{x}=\mathbf{0}\\
& ||\lambda \mathbf{x}||=|\lambda|\cdot ||\mathbf{x}||\\
& ||\mathbf{x}+\mathbf{y}||\leq ||\mathbf{x}||+||\mathbf{y}||
\end{align}
\end{definition}

%------------------------------------------------

\section{Notations}\index{Notations}

\begin{notation}
Given an open subset $G$ of $\mathbb{R}^n$, the set of functions $\varphi$ are:
\begin{enumerate}
\item Bounded support $G$;
\item Infinitely differentiable;
\end{enumerate}
a vector space is denoted by $\mathcal{D}(G)$. 
\end{notation}

%------------------------------------------------

\section{Remarks}\index{Remarks}

This is an example of a remark.

\begin{remark}
The concepts presented here are now in conventional employment in mathematics. Vector spaces are taken over the field $\mathbb{K}=\mathbb{R}$, however, established properties are easily extended to $\mathbb{K}=\mathbb{C}$.
\end{remark}

%------------------------------------------------

\section{Corollaries}\index{Corollaries}

This is an example of a corollary.

\begin{corollary}[Corollary name]
The concepts presented here are now in conventional employment in mathematics. Vector spaces are taken over the field $\mathbb{K}=\mathbb{R}$, however, established properties are easily extended to $\mathbb{K}=\mathbb{C}$.
\end{corollary}

%------------------------------------------------

\section{Propositions}\index{Propositions}

This is an example of propositions.

\subsection{Several equations}\index{Propositions!Several Equations}

\begin{proposition}[Proposition name]
It has the properties:
\begin{align}
& \big| ||\mathbf{x}|| - ||\mathbf{y}|| \big|\leq || \mathbf{x}- \mathbf{y}||\\
&  ||\sum_{i=1}^n\mathbf{x}_i||\leq \sum_{i=1}^n||\mathbf{x}_i||\quad\text{where $n$ is a finite integer}
\end{align}
\end{proposition}

\subsection{Single Line}\index{Propositions!Single Line}

\begin{proposition} 
Let $f,g\in L^2(G)$; if $\forall \varphi\in\mathcal{D}(G)$, $(f,\varphi)_0=(g,\varphi)_0$ then $f = g$. 
\end{proposition}

%------------------------------------------------

\section{Examples}\index{Examples}

This is an example of examples.

\subsection{Equation and Text}\index{Examples!Equation and Text}

\begin{example}
Let $G=\{x\in\mathbb{R}^2:|x|<3\}$ and denoted by: $x^0=(1,1)$; consider the function:
\begin{equation}
f(x)=\left\{\begin{aligned} & \mathrm{e}^{|x|} & & \text{si $|x-x^0|\leq 1/2$}\\
& 0 & & \text{si $|x-x^0|> 1/2$}\end{aligned}\right.
\end{equation}
The function $f$ has bounded support, we can take $A=\{x\in\mathbb{R}^2:|x-x^0|\leq 1/2+\epsilon\}$ for all $\epsilon\in\intoo{0}{5/2-\sqrt{2}}$.
\end{example}

\subsection{Paragraph of Text}\index{Examples!Paragraph of Text}

\begin{example}[Example name]
\lipsum[2]
\end{example}

%------------------------------------------------

\section{Exercises}\index{Exercises}

This is an example of an exercise.

\begin{exercise}
This is a good place to ask a question to test learning progress or further cement ideas into students' minds.
\end{exercise}

%------------------------------------------------

\section{Problems}\index{Problems}

\begin{problem}
What is the average airspeed velocity of an unladen swallow?
\end{problem}

%------------------------------------------------

\section{Vocabulary}\index{Vocabulary}

Define a word to improve a students' vocabulary.

\begin{vocabulary}[Word]
Definition of word.
\end{vocabulary}

%----------------------------------------------------------------------------------------
%	CHAPTER 3
%----------------------------------------------------------------------------------------

\chapterimage{chapter_head_1.pdf} % Chapter heading image

\chapter{Presenting Information}

\section{Table}\index{Table}

\begin{table}[h]
\centering
\begin{tabular}{l l l}
\toprule
\textbf{Treatments} & \textbf{Response 1} & \textbf{Response 2}\\
\midrule
Treatment 1 & 0.0003262 & 0.562 \\
Treatment 2 & 0.0015681 & 0.910 \\
Treatment 3 & 0.0009271 & 0.296 \\
\bottomrule
\end{tabular}
\caption{Table caption}
\end{table}

%------------------------------------------------

\section{Figure}\index{Figure}

\begin{figure}[h]
\centering\includegraphics[scale=0.5]{placeholder}
\caption{Figure caption}
\end{figure}


%----------------------------------------------------------------------------------------
% INTRODUCTION
%----------------------------------------------------------------------------------------
\chapter{Introduction}
\section{Introduction}\index{Introduction}

%----------------------------------------------------------------------------------------
% MODEL
%----------------------------------------------------------------------------------------
\chapter{Model}
\section{Model}\index{Model}

%----------------------------------------------------------------------------------------
% PARAMETERS ESTIMATION
%----------------------------------------------------------------------------------------
\chapter{Parameters Estimation}
\section{Parameters Estimation}\index{Parameters Estimation}


%----------------------------------------------------------------------------------------
% APPENDICES
%----------------------------------------------------------------------------------------
\chapter{Appendices}

\section{Introduction to copulas}\index{Introduction to copulas}

It is common to use correlation as a measure of dependency between random 
variables. In most cases, this measure does not fully reflect the structure 
of dependence between them. The mathematical concept that does reflect the 
structure of dependence between random variables, however, is the copula, 
which we define below \cite{McNeil:2005qf}.

%TODO: POSAR REFERENCIA
\begin{definition}[Copula]
A copula function, $C$, is a multivariate distribution defined on the 
unit hypercube $[0,1]^d$ with standard uniform marginals. 
More precisely,
\begin{displaymath}
C(u_1, \dots, u_d) = Pr\{U_1 \le u_1, \dots, U_d \le u_d\}
\end{displaymath}
where $U_i \sim \textrm{Uniform}(0,1) \textrm{ for } i = 1,\dots, d$.
\end{definition}

Sklar's theorem \cite{sklar:1959} states that any multivariate 
distribution with continuous marginals can be decomposed into the marginals and 
a copula that reflects the structure of dependence between them. Later, we will 
use this statement to define and simulate the t-Student copula.

%%TODO: POSAR REFERENCIA
\begin{theorem}[Sklar's theorem]
Let $F$ be an $d$-dimensional distribution function with margins $F_1,\dots,F_d$.
Then there exists an $d$-copula $C$ such that for all $x \in \mathbb{R}^d$,
\begin{displaymath}
F(x_1,\dots,x_d) = C(F_1(x_1),\dots,F_d(x_d))
\end{displaymath}
If $F_1,\dots,F_d$ are all continuous, the $C$ is unique; otherwise $C$ is uniquely
determined on $\textrm{Ran}F_1 \times \dots \times \textrm{Ran}F_d$.
Conversely, if $C$ is an $d$-copula and $F_1,\dots,F_d$ are distributions functions,
then the function $F$ defined above is an $d$-dimensional distribution function
with margins $F_1,\dots,F_d$.
\end{theorem}


\begin{corollary}[Copula of a multivariate distribution]
Let $X=(X_1, \dots, X_d)$ a random vector with a multivariate 
distribution $F$ and continuous marginals $F_1, \dots, F_n$. 
Then its copula is:
\begin{displaymath}
C(u_1,\dots,u_n) = F(F_1^{-1}(u_1), \dots, F_n^{-1}(u_n))
\end{displaymath}
\end{corollary}
\begin{proof}
This is a direct application of Sklar's theorem:
\begin{displaymath}
C(F_1(x_1), \dots, F_d(x_d)) = 
F(F_1^{-1}(F_1(x_1)), \dots, F_d^{-1}(F_d(x_d))) = 
F(x_1, \dots, x_d)
\end{displaymath}
\end{proof}


\begin{corollary}[Copula simulation]
Let $X=(X_1, \dots, X_d)$ a random vector with a multivariate 
distribution $F$ and continuous marginals $F_1, \dots, F_n$.
If we have a procedure to simulate $X$ then we can simulate 
its copula $C$ using:
\begin{displaymath}
(U_1, \dots, U_d) = (F_1(X_1), \dots, F_d(X_d))
\end{displaymath}
\end{corollary}
\begin{proof}
TODO: The probability integral transform
\end{proof}


\begin{corollary}[Multivariate distribution simulation]
Let $X=(X_1, \dots, X_d)$ a random vector with a copula $C$
and continuous marginals $F_1, \dots, F_n$. If we have a
procedure to simulate $C$ then we can simulate $X$ using:
\begin{displaymath}
(X_1, \dots, X_d) = (F_1^{-1}(U_1), \dots, F_d^{-1}(U_d))
\end{displaymath}
where $U_i$ are the copula components.
\end{corollary}
\begin{proof}
TODO
\end{proof}


\begin{definition}[Multivariate t-Student distribution]
The $d$-dimensional random vector $X=(X_1,\dots,X_d)$ is said to have a 
(non-singular) multivariate t-Student distribution with $\nu$ degrees of freedom, 
mean vector $\mu$ and positive-definite dispersion or scatter matrix $\Sigma$, 
denoted $t \sim t_d(\nu,\mu,\Sigma)$, if its density is given by
\begin{displaymath}
f(x)=\frac{\Gamma\left(\frac{\nu+d}{2}\right)}{\Gamma\left(\frac{\nu}{2}\right)\sqrt{(\pi \nu)^d |\Sigma|}}
\left(
1+ \frac{(x-\mu)^\top\Sigma^{-1}(x-\mu)}{\nu}
\right)^{-\frac{\nu+d}{2}}
\end{displaymath}
\noindent where $|\Sigma|$ represents the absolute value of the determinant of the matrix. 
\end{definition}

\begin{proposition}[Multivariate t-Student characterization]
A random vector $T \sim t_d(\nu,\mu,\Sigma)$ can be expresed as:
\begin{displaymath}
T \stackrel{d}{=} \mu + \sqrt{\frac{\nu}{V}}\cdot Z
\quad \textrm { where } Z \sim N(0,\Sigma) \textrm{ and } V \sim \chi_{\nu}^2
\end{displaymath}
\end{proposition}

\begin{proposition}[Multivariate t-Student marginals]
Let $X \sim t_d(\nu,\vec{0},\Sigma)$. Then, its i-th marginal is $X_i \sim t_1(\nu,0,R_{ii})$.
\end{proposition}

Note that the above proposition is also valid for the case $\nu=\infty$
corresponding to the multivariate Gaussian distribution.

We use the corollary XXX and the definition XXX to define the t-Student copula.

\begin{definition}[t-Student copula]
The t-Student copula, $C_{\nu,\Sigma}^d$, is the copula of the multivariate 
t-Student distribution $t_d(\nu,\mu,\Sigma)$.
\end{definition}

\begin{proposition}[t-Student copula invariance]
The copula of $t_d(\nu,\mu,\Sigma)$ is identical of that of $t_d(\nu,\vec{0},R)$
where $R$ is the correlation matrix implied by the dispersion matrix $\Sigma$.
\end{proposition}

\begin{proposition}[t-Student copula density]
The t-Student copula, $C_{\nu,R}^d$, where $R$ is a correlation matrix,
has the following distribution:
\begin{displaymath}
C_{\nu,R}^d(u_1, \dots, u_d) = 
\int_{-\infty}^{t_\nu^{-1}(u_1)} \dots \int_{-\infty}^{t_\nu^{-1}(u_d)} f(x) dx
\end{displaymath}
where $f(x)$ is the density function of $t_d(\nu,\vec{0},R)$, and $t_{\nu}^{-1}$ denotes the 
quantile function of the univariate distribution $t_1(\nu,0,1)$. 
The copula density is
\begin{displaymath}
\label{eq:density}
%\begin{array}{l}
c_{\nu,R}^d(u_1,\dots,u_d) = %\\
|R|^{-\frac{1}{2}} 
\displaystyle\frac{\Gamma{\left(\frac{\nu+d}{2}\right)}}{\Gamma{\left(\frac{\nu}{2}\right)}}
\displaystyle\left[ \frac{\Gamma{\left(\frac{\nu}{2}\right)}}{\Gamma{\left(\frac{\nu+1}{2}\right)}} \right]^d
\frac{\displaystyle\left( 1+\frac{\zeta' R^{-1} \zeta}{\nu}\right)^{-\frac{\nu+d}{2}}}{\displaystyle\prod_{i=1}^d \left( 1+\frac{\zeta_i^2}{\nu} \right)^{-\frac{\nu+1}{2}}}
%\end{array}
\end{displaymath}
\noindent
where $\zeta=(t_\nu^{-1}(u_1), \dots, t_\nu^{-1}(u_n))$ is the vector of the t-student 
univariate inverse distribution functions.
\end{proposition}
\begin{comment}
\begin{proof}
see ageeva
\end{proof}
\end{comment}


\section{Metropolis-Hastings}\index{Metropolis-Hastings}

\begin{algorithm}[Metropolis-Hastings]
Let us assume a target distribution $f(x)$ from which we wish to generate a 
sample of size $T$. The Metropolis-Hastings algorithm can be described by the 
following iterative steps; where $x^{(t)}$ is the vector of generated values 
in the $t$-th iteration of the algorithm:
\begin{enumerate}
\item Set initial values $x^{(0)}$
\item For $t=1,\dots,T$ repeat the following steps
\begin{itemize}
\item Set $x=x^{(t-1)}$
\item Let $x'$ a random value from the proposal distribution $q(x \to x’)=q(x'|x)$
\item Calculate the acceptance rate 
      $\alpha = \textrm{min}\left(1,\frac{f(x') \cdot q(x|x')}{f(x) \cdot q(x'|x)}\right)$
\item Update $x^{(t)}=\left\{
  \begin{array}{ll}
  x' & \textrm{ with probability } \alpha \\
  x^{(t-1)}  & \textrm{ with probability } 1-\alpha
  \end{array}\right.$ 
\end{itemize}
\end{enumerate}
\end{algorithm}

%TODO: list ergodic properties
%TODO: referenciar llibre WinBugs


Bayesian statistics differ from the classical statistical theory since all 
unknown parameters are considered as random variables.
We want calculate the distribution $f(\theta|y)$ of the parameters $\theta$ 
given the observed data $y$. According to the Bayes theorem, this can be 
written as:
\begin{displaymath}
f(\theta|y) = \frac{f(y|\theta) \cdot f(\theta)}{f(y)} \propto f(y|\theta) \cdot f(\theta)
\end{displaymath}

We call \emph{posterior distribution} to $f(\theta|y)$ and \emph{prior distribution} 
to $f(y)$. This prior distribution expresses the information available to the 
researcher before any data are involved in the statistical analysis. 
$f(y|\theta)$ is the probability of observing $y$ given $\theta$, and is also 
known as the likelihood.
\begin{displaymath}
f(y|\theta) = \prod_{i=1}^n f(y_i|\theta)
\end{displaymath}

The Metropolis-Hastings algorithm outlined above can be directly implemented to do
Bayesian inference by substituting $x$ by the unknow parameters, $\theta$, and 
the target distribution $f(x)$ by the posterior distribution $f(\theta|y)$. Then 
the acceptance rate becomes:
\begin{displaymath}
\alpha = \textrm{min}\left(1,\frac{f(\theta'|y) \cdot q(\theta|\theta')}{f(\theta|y) \cdot q(\theta'|\theta)}\right) = \textrm{min}\left(1,\frac{f(y|\theta') \cdot f(\theta') \cdot q(\theta|\theta')}{f(y|\theta)  \cdot f(\theta) \cdot q(\theta'|\theta)}\right)
\end{displaymath}

Random-walk Metropolis-Hastings is a special case with symmetric proposals
$q(\theta'|\theta) = q(\theta|\theta')$ where $q(\theta'|\theta) \equiv N(\theta,S)$. 
In this case the acceptance probability depends only on the posterior distribution
and, in the Bayesian inference Metropolis-Hastings case, becomes:
\begin{displaymath}
\alpha = 
\textrm{min}\left(1,\frac{f(y|\theta') \cdot f(\theta') \cdot q(\theta|\theta')}{f(y|\theta)  \cdot f(\theta) \cdot q(\theta'|\theta)}\right) = 
\textrm{min}\left(1,\frac{f(y|\theta') \cdot f(\theta')}{f(y|\theta)  \cdot f(\theta)}\right)
\end{displaymath}

\begin{comment}
  A continuación se adapta el algoritmo descrito. Como proposal distribution q se usa q(θ’|θ)=N(θ,σ) que tiene la propiedad que q(θ’|θ)= q(θ|θ’) y simplifica la expresión α. También se modifica para operar con parámetros de dimensión d.
\end{comment}

\section{Constructing marginals}\index{Constructing marginals}

\begin{comment}
\end{comment}

%----------------------------------------------------------------------------------------
%	BIBLIOGRAPHY
%----------------------------------------------------------------------------------------
\chapter*{Bibliography}
\addcontentsline{toc}{chapter}{\textcolor{ocre}{Bibliography}}
\section*{Books}
\addcontentsline{toc}{section}{Books}
\printbibliography[heading=bibempty,type=book]
\section*{Articles}
\addcontentsline{toc}{section}{Articles}
\printbibliography[heading=bibempty,type=article]

%----------------------------------------------------------------------------------------
%	INDEX
%----------------------------------------------------------------------------------------
\cleardoublepage
\setlength{\columnsep}{0.75cm}
\addcontentsline{toc}{chapter}{\textcolor{ocre}{Index}}
\printindex

%----------------------------------------------------------------------------------------

\end{document}
